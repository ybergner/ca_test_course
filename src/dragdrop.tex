%%%%%%%%%%%%%%%%%%%%%%%%%%%%%%%%%%%%%%%%

\begin{edXchapter}{Advanced Examples}

\begin{edXsection}{Drag and drop input}

\begin{edXvertical}

\begin{edXproblem}{Approximate energy level diagram}{attempts=10}

Use your knowledge of the size of the Lamb shift, and the scale of fine
structure splittings, to properly label the following energy level diagram.

\edXinclude{XML/ps4-hydrogen-dnd.xml}

\begin{edXsolution}

\begin{itemize}

\item The fine structure splitting raises the four $2P_{3/2}$ states about
10 GHz above the two $2P_{1/2}$ and two $2S_{1/2}$ states.

\item The Lamb shift raises $2S_{1/2}$ above $2P_{1/2}$ by 1.06 GHz.

\end{itemize}

\end{edXsolution}

\edXgitlink{\giturl}{Source TeX}
\end{edXproblem}


\end{edXvertical}

%%%%%%%%%%%%%%%%%%%%

\begin{edXvertical}

\begin{edXproblem}{Drag and drop problem created via latex2dnd}{attempts=1000}

The \href{https://github.com/mitocw/latex2dnd}{latex2dnd} package can
be used to generate edX
\href{http://edx.readthedocs.org/projects/devdata/en/latest/course_data_formats/drag_and_drop/drag_and_drop_input.html}{drag-and-drop}
problems.  These problems are very convenient for complex equations,
or for labeling of graphical images:

\begin{edXshowhide}{dndex}{LaTeX code for quadratic polynomial example}
Download \href{https://github.com/mitocw/latex2dnd/blob/master/latex2dnd/testtex/quadratic.tex}{from github}:
\begin{verbatim}
\documentclass{article}
\input{latex2dnd}

\begin{document}

% define drag-drop labels

\DDlabel{term1}{$-b$}
\DDlabel{term1p}{$+b$}
\DDlabel{term2}{$b^2$}
\DDlabel{dubexp}{$b^{2^\alpha}$}
\DDlabel{dubsub}{$b_{2_\alpha}$}
\DDlabel{fac}{$-4ac$}
\DDlabel{facp}{$+4ac$}
\DDlabel{ta}{$+2a$}
\DDlabel{tam}{$-2a$}

% shorthand macro to make all boxes the same size (6 by 4)
\newcommand\DDB[2]{\DDbox{#1}{6ex}{4ex}{#2}}

% the formula with boxes
$$\lambda = \frac{\DDB{1}{term1}\pm \sqrt{\DDB{2}{term2}\DDB{3}{fac}}}{\DDB{4}{ta}}$$

% output labels, with fixed box heights
\writeDDlabels[4.3ex]

\end{document}
\end{verbatim}
\end{edXshowhide}

Complete this to give the correct equation for the roots of a
quadratic polynomial $ax^2 + bx + c$:

\edXinclude{dnd/quadratic_dnd.xml}

\edXgitlink{\giturl}{Source TeX}
\end{edXproblem}

%%%%%%%%%%

\begin{edXproblem}{Drag and drop problem with formula grading}{attempts=1000}

This is an example of a problem created using
\href{https://github.com/mitocw/latex2dnd}{latex2dnd} which is graded
as a mathematical formula.

Using formula grading allows any input which is mathematically
equivalent to the expected answer to be accepted as being correct.

\begin{edXshowhide}{dndex2}{LaTeX code for drag-and-drop graded as formula example}
Download \href{https://github.com/mitocw/latex2dnd/blob/master/latex2dnd/testtex/quadratic.tex}{from github}:
\begin{verbatim}
\documentclass{article}
\input{latex2dnd}

%%%%%%%%%%%%%%%%%%%%%%%%%%%%%%%%%%%%%%%%%%%%%%%%%%%%%%%%%%%%%%%%%%%%%%%%%%%%%

\begin{document}

% define drag-drop labels
\DDlabel{G}{$G$}
\DDlabel{m1}{$m_1$}
\DDlabel{m2}{$m_2$}
\DDlabel{d}{$d$}
\DDlabel{d2}{$d^2$}

% shorthand macro to make all boxes the same size (6 by 4)
\newcommand\DDB[2]{\DDbox{#1}{6ex}{4ex}{#2}}

% the formula with boxes
$$F = \frac{\DDB{1}{G} \DDB{2}{m1}\DDB{3}{m2}}{\DDB{4}{d2}}$$

% the formula to use for correctness checking, the samples, and the expected answer
% place target_id names inside square brackets
\DDformula{ [1] * [2] * [3] / [4] }{ G,m_1,m_2,d@1,1,1,1:20,20,20,20\#40 }{G*m_1*m_2/d^2}{Try checking units}

% output labels, with fixed box heights
\writeDDlabels[4.3ex]

\end{document}
\end{verbatim}
\end{edXshowhide}

Give Newton's formula for the gravitational force between two bodies:

\edXinclude{dnd/gravity_dnd.xml}

\edXgitlink{\giturl}{Source TeX}
\end{edXproblem}

%%%%%%%%%%

\end{edXvertical}

\end{edXsection}

%%%%%%%%%%%%%%%%%%%%%%%%%%%%%%%%%%%%%%%%%%%%%%%%%%%%%%%%%%%%%%%%%%%%%%%%%%%%%

\begin{edXsection}{Graphical input using javascript}

\begin{edXvertical}

\begin{edXproblem}{Graphical input via jsxgraph}{attempts=1000}

The \href{http://edx.readthedocs.org/projects/devdata/en/latest/course_data_formats/jsinput.html}{jsinput} framework provides a very convenient way to
employ virtually arbitrary javascript code as a user interface for
input into edX auto-graded problems.  

This is an example which uses sliders and graphs from
\href{http://jsxgraph.uni-bayreuth.de/}{jsxgraph}, a nice javascript
math function plotting package, as input to a physics problem.

     Identify one such field-independent transition in the figure
     below, by doing the following:
     \begin{itemize}
     \item Use the slider to select a value of $x$
     \item Choose two different states (A...H) for the transition
     \item Click ``Check'' to check your answer
     \end{itemize}

\edXincludepy{XML/test_findep1.py}

% this javascript forces the saved jsinput state to be reloaded after 1.4 seconds
\edXxml{<script type="text/javascript">setTimeout(function(){ JSInput.jsinputConstructor(\$('section.jsinput'))}, 1400);</script>}

\edXabox{expect="" type="jsinput" cfn="test_findep" 
  width="650"
  height="555"
  gradefn="getinput"
  get_statefn="getstate"
  set_statefn="setstate"
  html_file="/static/html/ps3plot_btran1.html"
 }%

\edXgitlink{\giturl}{Source TeX}
\end{edXproblem}

\end{edXvertical}

\end{edXsection}

%%%%%%%%%%%%%%%%%%%%%%%%%%%%%%%%%%%%%%%%%%%%%%%%%%%%%%%%%%%%%%%%%%%%%%%%%%%%%

\end{edXchapter}
